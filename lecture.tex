
\documentclass[a4paper, 12pt]{article}
%\usepackage{mathtext}
\usepackage{cmap}
\usepackage[english, russian]{babel}
\usepackage[T2A]{fontenc}
\usepackage[utf8]{inputenc}
\usepackage[left=2cm, right=1.5cm, top=2cm, bottom=2cm]{geometry}
\usepackage{amsmath}
\usepackage{amssymb}
\usepackage{etoolbox}
\usepackage{amsthm}
\usepackage{amsfonts}
\usepackage{mathtools}
%\usepackage{indentfirst}
\usepackage{soulutf8}


\usepackage{tikz,amstext}
\newlength{\tempheight}
\newcommand{\Let}[0]{%
\mathbin{\text{\settoheight{\tempheight}{\mathstrut}\raisebox{0.5\pgflinewidth}{%
\tikz[baseline,line cap=round,line join=round] \draw (0,0) --++ (0.4em,0) --++ (0,1.5ex) --++ (-0.4em,0);%
}}}}

\newcommand{\R}{\mathbb R}
\newcommand{\Q}{\mathbb Q}
\renewcommand{\phi}{\varphi}
\renewcommand{\epsilon}{\varepsilon}
\newcommand{\aug}{\fboxsep=-\fboxrule\!\!\!\fbox{\strut}\!\!\!}

\theoremstyle{definition}
\newtheorem*{definition}{Определение}
\newtheorem*{theorem}{Теорема}%[section]cle
\newtheorem*{consequense}{Следствие}%[theorem]
\newtheorem*{lemma}{Лемма}
\newtheorem*{subtheorem}{Утверждение}
\newtheorem*{remark}{Замечание}
\newtheorem*{example}{Примеры}
\newtheorem*{lalala}{Упражнение}
\newtheorem*{algorithm}{Алгоритм}


\usepackage[russian]{babel}
\addto\captionsenglish{% Replace "english" with the language you use
  \renewcommand{\contentsname}%
    {Содержание}%
}

\usepackage{titlesec}
\titleformat{\section}{\LARGE \bfseries}{\thesection}{1em}{}
\titleformat{\subsection}{\Large\bfseries}{\thesubsection}{1em}{}
\titleformat{\subsubsection}{\large\bfseries}{\thesubsubsection}{1em}{}

\usepackage{hyperref}
\usepackage{xcolor}
% Цвета для гиперссылок
\definecolor{linkcolor}{HTML}{225ae2} % цвет ссылок
\definecolor{urlcolor}{HTML}{225ae2} % цвет гиперссылок
\hypersetup{
    pdfstartview=FitH, 
    linkcolor=linkcolor,
    urlcolor=urlcolor,
    colorlinks=true
}

\title{\textbf{Алгебра. 1 семестр, О.В.Куликова}}
\author{Вячеслав Молчанов, 108 группа}

\begin{document}
  \maketitle
  \newpage
  \tableofcontents
  \fontsize{14pt}{20pt}\selectfont
  \newpage
  \section{Система линейных уравнений}
  \subsection{Матрица. Основные понятия}

  \begin{definition}
    Матрица $A$ размера $m\times n$ прямоугольная таблица с $m$ строками и $n$ столбцами
    $$ A= \begin{pmatrix}
      a_{11} && a_{12} && \dots && a_{1n}\\
      a_{21} && a_{22} && \dots && a_{2n}\\
      \vdots && \null && \null && \vdots\\
      a_{m1} && a_{m2} && \dots && a_{mn}
    \end{pmatrix}$$ \\
    $a_{ij}$ - элемент матрицы и индыксы:

    \begin{itemize}
      \item $i$ - номер строками
      \item $j$ - номер столбца
    \end{itemize}
    
    $M_{m\times n}(\R)$ - Множество всех матриц размера $m\times n$ с элементами из $\R$
    \end{definition}
    Матрица $m\times 1$ называется столбцом:
    $$ A= 
    \begin{pmatrix}
      a_{11} \\
      a_{21} \\
      \vdots \\
      a_{m1} 
    \end{pmatrix} $$

    Если $A=(a_{ij})$ - крадратная, $a_{ij} = 0\ \forall i \neq j$, то $A$ называется диальнольной. \\
    $$ A =
    \begin{pmatrix}
      a_{11} && \null && \null && 0 \\
      \null && a_{22} && \null && \null \\
      \null && \null && \ddots && \null \\
      0 && \null && \null && a_{nn} 
    \end{pmatrix} $$ \\
 

    Если $A$ - диальнольная и $a_{ij}$ = 1, то $A$ называется единичной.
    $$ A =
    \begin{pmatrix}
      1 && \null && \null && 0 \\
      \null && 1 && \null && \null \\
      \null && \null && \ddots && \null \\
      0 && \null && \null && 1 
    \end{pmatrix} $$ \\

    \newpage

    Если $A$ - квадратная, то
    \begin{itemize}
      \item $ A =\begin{pmatrix}
        a_{11} && \null && \null \\
        \null && \ddots && \null \\
        \null && \null && a_{nn} 
      \end{pmatrix} $ главная диагональ
      \item $ A =
      \begin{pmatrix}
        \null && \null && a_{n1} \\
        \null && \dots && \null \\
        a_{n1}  && \null && \null
      \end{pmatrix} $ побочная диагональ
    \end{itemize}
    
    \begin{definition}
      Если $A$ - размера $m\times n$, $a_{ij} = 0\ \forall i,j$, то $A$ называется нулевой.
    \end{definition}

    \subsection{Система линейных (алгебраческих) уравнений}
    $(*)\begin{cases}
      a_{11}x_1 + ... + a_{1n}x_n = b_1 \\ 
      a_{21}x_2 + ... + a_{2n}x_n = b_2 \\
      \vdots \\
      a_{n1}x_1 + ... + a_{nn}x_n = b_n
    \end{cases}$ \\$\\$
    где $a_{ij}, b \in \R, x_1,... ,x_n$ - неизвестные.\\
    $$A = \begin{pmatrix}
      a_{11} && \dots && a_{1n} \\
      \vdots && \null && \vdots \\
      a_{n1} && \dots && a_{nn} 
    \end{pmatrix} \hspace{30pt} B = \begin{pmatrix}
      a_{11} \\
      \vdots \\
      b_{n}
    \end{pmatrix}$$
    $A$ - матрица коэфициентов, $a_{ij}$ называется коэфициентом СЛУ.\\
    $B$ - столбец свобоных членов, $b_{j}$ - свободный член.
    \begin{definition}
      Расширенная матрица $\underset{m\times (n+1)}{(A|B)}$. Набор чисел $x_1^0,...,x_n^0 \in \R$ называется решением системы $(*)$, если подстановка этих чисел вместо неизвестных в $(*)$ дает тождество в каждом уравнении. $(x_i^0\longleftrightarrow x)$ 
    \end{definition}
    Решить систему - это найит все решения системы. Любое конткретное решение называется частным.
    \begin{definition}
      Если СЛУ имеет решение, то она называется совместной, иначе несовместной. 
    \end{definition}  
    \begin{definition}
      Совместная система, имеющая одно решение, называется определенной, иначе неопределенной (более одного решения).
    \end{definition}  
    \newpage
    \subsection{Элементарные преобразования над СЛУ}
    \begin{enumerate}
      \item Прибавить к одному уравнению другое уравнение, умноженное на число $\lambda \in \R$
      \item Поменять местами два уравнения
      \item Умножить уравнение на ненулевое число $\mu \in \R$
    \end{enumerate}
    \begin{subtheorem}
      Эти преобразования обратимы.
    \end{subtheorem}
    \begin{definition}
      Две системы линеных уравнений называются эквивалентными, если их множества решений совпадают.
    \end{definition}
    \begin{subtheorem}
      Если одна СЛУ получена из другой СЛУ с помощью конечного числа элементарных преобразований, то эти системы эквивалентны.
    \end{subtheorem}
    \begin{proof}
      $ \\ \Longrightarrow$  
      Достаточно доказать, что если $AX = B$  и $\tilde{A} X = \tilde{B}$ получены с помощью одного элементарного преобразования, то она эквивалентны.
      Пусть $x_1^0,...,x_n^0$ - произвольное решение $AX = B$.
      Докажем, что $x_1^0,...,x_n^0$ является решением системы $\tilde{A} X = \tilde{B}$.
      Для 2 пункта очевидно. Для 3: предположим, что 
      $$a_{i1}x_1 +...+ a_{in} = b_{i} \text{ в } AX = B$$
      $$ (\mu a_{i1})x_1 +...+(\mu a_{in})=x_{1}n = \mu b_i \text{ в } \tilde{A} X = \tilde{B} $$ 
      Если $a_{i1}x_1 +...+ a_{in} = b_{i}$, то $ (\mu a_{i1})x_1 +...+(\mu a_{in})=x_{1}n = \mu b_i$ \\
      Остальные уравнения такие жн $\mu (a_{i1}x_1 +...+ a_{in}) = (\mu)b_i$ \\ 
      1 пункт: Д/З \\
      Т.о. множество решений $AX = B$ включено в множество решений $\tilde{A} X = \tilde{B}$. \\
      $\Longleftarrow$ В обратную сторону аналогично (для доказательства эквивалентности), используя обратимость элементарных преобразований.
    \end{proof}
    Мораль в том, что мы можем работать с расширенной матрицей $(A|B)$.
    \newpage
    \subsection{Элементарные преобразования над матрицами}
    \textbf{Элементарные преобразования над строками:} \\
    $$A =\begin{pmatrix}
      \overline{a_1} \\
      \overline{a_2} \\
      \vdots \\
      \overline{a_i}
    \end{pmatrix}, \text{ где } \overline{a_i} - \text{строка}$$
    \begin{itemize}
      \item ЭП1: $\overline{a_i} \to \overline{a_i} + \lambda \overline{a_i}$
      \item ЭП2: $\overline{a_i} \longleftrightarrow   \overline{a_j}$
      \item ЭП3: $\overline{a_i} \to \mu \overline{a_i},\ \mu \neq 0$
  \end{itemize}
  \begin{definition}
    Лидер строки (ведущий элемент) - это 1-й ненулевой элемент слева. \\
    \textbf{Пример:} $(0, 0, \underbrace{3}_{\text{лидер}}, 4, 5, 0, 0, 7)$
  \end{definition}
  \begin{definition}
    Матрица $A$ размера $m\times n$ называется ступенчатой, если 
    \begin{enumerate}
      \item Номера лидеров ненулевых строк строго возрастают с увеличением номера строки.
      \item Все нулевые строки стоят внизу (в конце).
    \end{enumerate}
  \end{definition}
  \begin{theorem}
    Любую матрица $A$ размера $m\times n$ за конечное число элементарных преобразований над строками можно привести к ступенчатому виду.
  \end{theorem}
  \begin{proof} Индукция по $n$: \\
    Если $A$ - нелувая, то $A$ - ступенчатого вида. Если $A \neq 0$ : найдем первый ненулевой столбец (начиная слева). Пусть $j$ - номер первого ненулевого столбца. Пусть $a_{ij} \neq 0$: 
    $$A =\begin{pmatrix}
      0 && 0 && \null && \null  \\
      \vdots && \vdots && \null  \\
      \null && \null && a_{ij} && \null \\
      \vdots && \vdots && \null  \\
      0 && 0 && \null && \null  \\
    \end{pmatrix}$$ 
    \newpage
    Меняем 1-ю и $i$-ю строку местами и получаем, что $a_{ij}$ стал лидером первой строки. Считаем, что сразу $a_{1j} \neq 0$:
    $$A = \begin{pmatrix}
      0 && 0 && a_{ij} && * \\
      \vdots && \vdots && * && *  \\
      \null && \null && \vdots && \null \\
      \vdots && \vdots && \vdots && \null  \\
      0 && 0 && \vdots && \null  \\  
    \end{pmatrix} $$ \\
    Вычитаем из кажкой $k$-й строки, начиная со 2-ой, 1-ю строку, умноженную на число $\frac{a_{kj}}{a_{1j}}$. Получает вид: 
    $$\tilde{A} =\begin{pmatrix}
      0 && 0 && \vline && * && \null \\
      \vdots && \vdots && \vline && * && * \\
      \null && \null && \vline && \vdots && \null \\
      \vdots && \vdots && \vline && \vdots && \null  \\
      0 && 0 && \vline && \vdots && \null  \\
    \end{pmatrix}$$ \\
    К правой части матрицы применяем индукцию и проводим матрицу к ступенчатому виду.
  \end{proof}
  \begin{remark}
    Этот метод называется  методом Гауса.
  \end{remark}

  \subsection{Решение СЛУ методом Гауса}
  $\begin{cases}
    a_{11}x_1 + ... + a_{1n}x_n = b_1 \\ 
    a_{21}x_2 + ... + a_{2n}x_n = b_2 \\
    \vdots \\
    a_{m1}x_1 + ... + a_{mn}x_n = b_m
  \end{cases}$ \\$\\$

  Элементарные преобразования над $AX=B$ $\Longleftrightarrow$ элементарные преобразования над $(A|B)$. 

  СЛУ $AX=B$ ступенчатая $\Longrightarrow $ $(A|B)$ имеет ступенчатый вид. \\ 
  \newpage

  \begin{subtheorem}
    Решение СЛУ ступенчаного вида. \\
  \end{subtheorem}  
  Пусть $AX=B$ - ступенчатая
  $$(A|B)=\begin{pmatrix}
    a_{11} & \null & \null & \null & \vline & b_1 \\
    \null & a_{22} & \null & \null & \vline & \vdots \\
    \null & \null & \ddots & \null & \vline & \vdots \\
    \null & \null & \null & a_{sn} & \vline & b_s \\
    \null & \null & \null & \vdots & \vline & \vdots \\
    0 & \cdots & \cdots & 0 & \vline & b_{\widetilde{s}}
  \end{pmatrix}$$ \\
  $\widetilde{s}$ - ненулевые строки расширенной матрицы \\
  s - число ненулевых строк 

  $\widetilde{s}=\left[
    \begin{gathered}
      s \\
      s+1
    \end{gathered}
  \right.$


  \begin{itemize}
    \item[1 случай:]
    $\widetilde{s} \neq s \ (\widetilde{s}=s+1)$ \\ 
    Рассмотрим последнюю ненулевую строку: \\
    $$\begin{pmatrix}
      a_{11} & \null & \null & \null & \vline & b_1 \\
      \null & a_{22} & \null & \null & \vline & \vdots \\
      \null & \null & \ddots & \null & \vline & \vdots \\
      \null & \null & \null & a_{sn} & \vline & b_s \\
      0 & \cdots & \cdots & 0 & \vline & b_{s+1}
    \end{pmatrix}$$ \\
    $0x_1+...+0x_n=b_s+1$ 
    $\Longrightarrow$ решение у этого уравнения нет 
    $\Longrightarrow$ СЛУ не имеет решения, т.е. несовместнаю. \\
    Далее $\widetilde{s}=s$\\
    Заметим, что $\widetilde{s}=s\leq n$ (n-колличество столбов)
    \item[2 случай:] $\widetilde{s}=s=n$  
    $$\left\{ \begin{aligned}
      a_{11} x_1 + a_{12} x_2+ \dots + a_{1n} x_n = b_1 \\
      a_{22} x_2 + \dots + a_{1n} x_n = b_2 \\ 
      \ddots \ \ \ \ \ \ \ \ \ \ \ \ \ \ \ \ \vdots \ \\
      a_{nn} x_n = b_n
    \end{aligned}
    \right.$$

    Такая СЛУ называются строго треугольной

    Из n-го уравнения однозначно находится $x_n = \frac{b_n}{a_{nn}}$
    Подставляем во все оставшиеся уравнения $x_n = \frac{b_n}{a_{nn}}$ $\Longrightarrow$ исключаем $x_n$. Получаем строго треугольную систему с меньшим колличество неизвестных.  \\
    Далее из (n-1)-го уравнения  находим $x_{n-1}$ и т.д. $\Longrightarrow$ СЛУ имеет единственное решение т.е. является определенной.

    \item[3 случай:] $\widetilde{s}<n$ 
  $$\begin{pmatrix}
    0 & \null & 0 & |\underline{a_{1k_1}} & \ast & \cdots & \cdots & \ast & \vline & \ast  \\
    0 & \null & 0 & 0 & |\underline{a_{2k_2}} & \ast & \cdots & \ast & \vline & \ast \\
    \null & \null & \null & \null & \null & \ddots & \null & \null & \vline & \vdots
  \end{pmatrix}$$ 

  $a_{1k_{1}},...,a_{sk_{s}}$  - лидеры; \\
  $x_{k_{1}},...,x_{k_{s}}$ - главные неизвестные (неизвестные соответствуют лидерам) \\
  Оставшиеся неизвестные назовем свободными. \\
  Перекинем в правую часть СЛУ слагаемые, соответствующие свободным неизвестным 
  $\Longrightarrow$ получаем относительно главных неизвестных строго треугольную СЛУ. \\
  Как в случае 2 однозначно выражается главнае неизвестные через свободные
  $\Longrightarrow$ с точностью до нумерации получаем: \\
  $$\begin{cases}
    x_1 = c_{1,s+1}x_{s+1} + \dots + c_{1n}x_n+d_1 \\
    \vdots \\
    x_s = c_{s,s+1}x_{s+1} + \dots + c_{sn}x_n+d_s
  \end{cases}$$   
  
  Это выражение называется общим решение системы. Подставляя вместо свободных неизвестных конткретное число из $\R$, получаем значение для главных. \\
  $\Longrightarrow$ получаем все решения СЛУ\\
  Если СЛУ имеет > 1 решения - такое СЛУ называется неопределенным. 
  \end{itemize}
  $$\begin{matrix}
    \null &&& \null && \ \ \text{\ \ \ \ СЛУ} \\
    \null && \null && \swarrow && \searrow \\
    &&& \widetilde{s} \neq s && \null &&\widetilde{s} = s\\
    \null &&& \text{несовместна} && \null && \text{совместна} \\
    \null && \null && \null && \swarrow \null && \searrow \\
    \null && \null && \null && \widetilde{s} = s = n \null && \widetilde{s} = s \leq n \\
    \null && \null && \null && \text{определенна} \null && \text{неопределенна}
  \end{matrix}$$
  \\
  \begin{algorithm}
    $AX=B \longmapsto (A|B) \thicksim (A_{c}|B_{c})\longmapsto A_cX=B_c$
  \end{algorithm} 
  \newpage
  \begin{definition} 
    Матрица $A$ имеет улучшенный ступенчатый вид, если выполнены следующие условия:
    \begin{enumerate}
      \item $A$ - ступенчатого вида
      \item Все лидеры равны 1
      \item В каждом столбце, где есть лидер $\neq 0$ , все элементы равны 0 
    \end{enumerate}
  \end{definition}  

  \begin{subtheorem}
    Любую матрицу $A$  можно привести к улучшенному ступенчатому виду с помощью элементарных преобразований 
  \end{subtheorem} 
  \begin{proof}
    Т.к. любую матрицу можно привести к ступенчатому виду $\Longrightarrow$ будем считать, что $A$ - ступенчатая. \\
    Рассмотрим последний лидер $a_{sk_s}$. Если $a_{sk_s} \neq 1$, то s-ю строку делим на $a_{sk_s}$ и получаем, что $\widetilde{a_{sk_s}}=1$. \\ Далее из всех строк вычитаем первую, умноженную на $a_{ik_s} \Longrightarrow \widetilde{a_{ik_s}}  =0$ и т.д. 
  \end{proof} 

  \begin{definition}
    СЛУ $AX=B$ называется однородной, если $B=0$, т.е. все свободные члены ненулевые.  
  \end{definition} 
  \begin{subtheorem}
    Однородная система всегда совместна.
  \end{subtheorem} 
  \begin{proof}
    $AX=0$ всегда имеет решение $x_1=0,...,x_n=0$ (тривиальное решение)
  \end{proof} 
  \begin{consequense}
    Однородная СЛУ, в которой число уравнений $<$ числа неизвестных, имеет нетривиальное решение.  
  \end{consequense} 
  \begin{proof}
    (в обозначениях из метода Гаусса)\\
    Т.к. система совместна (т.к. $B=0$), то $s=\widetilde{s}$ \\
    С другой стороны $s=\overline{s} \leq$ число исходных уравнений $<$ n $\Longrightarrow s=\widetilde{s} < n \Longrightarrow$ СЛУ неопределенна $\Longrightarrow \exists$ более одного решения $\Longrightarrow \exists$ нетривиальное решение.     
  \end{proof} 
  \newpage
  \section{Векторные пространства}
  Мы рассматриваем векторные пространства над полем $\R$.
  \begin{definition}
    Векторным пространством над $\R$ называют множество элементов $V$, на котором введены операции сложения и умножения на числа из $\R$:
    \begin{enumerate}
      \item $ \forall x,y \in V \Longrightarrow x+y=z \in V$
      \item $\forall \lambda \in \R, \forall x \in V \Longrightarrow \lambda x = w \in V$ 
    \end{enumerate}
    Удовлетворяет следующими свойствами:
    \begin{enumerate}
      \item x+y = y+x (коммутативность)
      \item (x+y)+z = x+(y+z) (ассоциативность)
      \item $\exists 0 \in V: \forall x \in V: x+0 = 0+x = x$ (нейтральный элемент относильно сложения)
      \item $\forall x \in V: \exists x^{\prime}: x + x^{\prime} = 0$ (противоположный элемент)
      \item $\forall \lambda \in \R, \forall x,y \in V: \lambda (x+y) = \lambda x + \lambda y$ (дистрибутивность сложения отностильно умножения)
      \item $\forall \lambda, \mu \in \R, \forall x \in V: (\lambda+\mu)x = \lambda x + \mu x $ (дистрибутивность умножения отностильно сложения)
      \item $\forall \lambda, \mu \in \R, \forall x \in V: \lambda(\mu x) = (\lambda \mu) x $ (ассоциативность умножения)
      \item $\forall x \in V: 1 \cdot x = x$ (нейтральный элемент относильно умножения)
    \end{enumerate}
  \end{definition} 

  \begin{definition}
    Любой элемент векторного пространства называется вектором
  \end{definition} 

  \textbf{Примеры векторных пространств:} 
    \begin{enumerate} 
      \item $V^2$ Геометрические векторы на плоскости
      \item $\R^n$ = $\{ {(a_1,...,a_n) | a_i \in \R} \}$ - арифметические векторы 
    \end{enumerate}
    "$+$": $(a_1,...,a_n)$ + $(b_1,...,b_n)$ = $(a_1+b_1,...,a_n+b_n)$ \\
    "$\times$": $(a_1,...,a_n) \times \lambda$ = $(a_1\lambda,...,a_n\lambda)$ 
  \begin{lalala}
    Проверьте, что $\R^n$ (арифметическое пространство строк) с этими операцияими является векторным пространством. 
  \end{lalala}   

\end{document}


